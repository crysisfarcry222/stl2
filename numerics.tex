\rSec0[numerics]{Numerics library}

\synopsis{Header \tcode{<experimental/ranges/random>} synopsis}

\begin{codeblock}
namespace std { namespace experimental { namespace ranges { inline namespace v1 {
  template <class G>
  concept bool UniformRandomNumberGenerator = @\seebelow@;
}}}}
\end{codeblock}

%%%%%%%%%%%%%%%%%%%%%%%%%%%%%%%%%%%%%%%%%%%%%%%%%%%%%%%%%%%%%%%%%%%%%%
% Uniform Random Number Generator requirements:

\rSec1[rand.req.urng]{Uniform random number generator requirements}%
\indextext{uniform random number generator!requirements|(}%
\indextext{requirements!uniform random number generator|(}

\begin{codeblock}
// defined in <experimental/ranges/random>

namespace std { namespace experimental { namespace ranges { inline namespace v1 {
  template <class G>
  concept bool UniformRandomNumberGenerator =
    requires(G g) {
      { g() } -> UnsignedIntegral; // not required to be equality preserving
      { G::min() } -> Same<result_of_t<G&()>>;
      { G::max() } -> Same<result_of_t<G&()>>;
    };
}}}}
\end{codeblock}

\pnum
A \techterm{uniform random number generator}
\tcode{g} of type \tcode{G}
is a function object
returning unsigned integer values
such that each value
in the range of possible results
has (ideally) equal probability
of being returned.
\enternote
 The degree to which \tcode{g}'s results
 approximate the ideal
 is often determined statistically.
\exitnote

\pnum
Let \tcode{g} be any object of type \tcode{G}. Then
\tcode{UniformRandomNumberGenerator<G>} is satisfied if and only if

\begin{itemize}
\item Both \tcode{G::min()} and \tcode{G::max()} are constant expressions~(\cxxref{expr.const}).
\item \tcode{G::min() < G::max()}.
\item \tcode{G::min() <= g()}.
\item \tcode{g() <= G::max()}.
\item \tcode{g()} has amortized constant complexity.
\end{itemize}
