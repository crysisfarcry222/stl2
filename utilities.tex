%!TEX root = std.tex
\setcounter{chapter}{19}
\rSec0[utilities]{General utilities library}

\setcounter{section}{1}
\rSec1[utility]{Utility components}

\ednote{``\tcode{swap}'' is a customization point found by Argument-Dependent Lookup. Due to the nature of
ADL, the \tcode{swap} in namespace \tcode{std} would be considered if either argument has namespace \tcode{std}
as an associated type. If \tcode{std::swap} remains unconstrained, then the \tcode{Swappable} concept will
erroneously return true for some types that are not in fact swappable. Below, we suggest adding constraints
directly to the overloads of \tcode{swap} in namespace \tcode{std} rather than defining new constrained
overloads of \tcode{swap} in namespace \tcode{std::experimental::ranges_v1}. The precise mechanism by which
\tcode{swap} is constrained such that it is usable in build environments lacking the concepts language feature
is unspecified.}

\synopsis{Header \tcode{<utility>} synopsis}

\pnum
Change the \tcode{<utility>} synopsis as follows:

\begin{codeblock}
@\added{namespace std \{}@
  // ... as before ..

  // \ref{utility.swap}, swap:
  template<class T>
    @\added{requires experimental::ranges_v1::Movable<T>()}@
  void swap(T& a, T& b) noexcept(@\seebelow@);

  template <class T, size_t N>
    @\added{requires requires (remove_all_extents_t<T>\& t) \{ swap(t, t); \}}@
  void swap(T (&a)[N], T (&b)[N]) noexcept(noexcept(swap(*a, *b)));}

  // ... as before ..
@\added{\}}@
\end{codeblock}

{\color{addclr}
\synopsis{Header \tcode{<experimental/ranges_v1/utility>} synopsis}
}

\begin{codeblock}
@\added{namespace std \{ namespace experimental \{ namespace ranges_v1 \{}@
  @\added{// \ref{utility.swap}, swap:}@
  @\added{using std::swap;}@

  // \ref{utility.exchange}, exchange:
  template <@\removed{class}\added{Movable}@ T, class U=T>
    @\added{requires Assignable<T\&, U>()}@
  T exchange(T& obj, U&& new_val);
\end{codeblock}
{\color{addclr}
\begin{codeblock}
  // \ref{taggedtup.tagged}, struct with named accessors
  template <class Base, class... Types> struct tagged;

  // \ref{tagged.special}, tagged specialized algorithms
  template <class Base, class... Tags>
    requires Swappable<Base&>()
  void swap(tagged<Base, Tags...>& x, tagged<Base, Tags...>& y) noexcept(@\seebelow@);

  // \ref{tagged.pairs}, tagged pairs
  template <class T1, class T2> using tagged_pair = @\seebelow@;

  template <class Tag1, class Tag2, class T1, class T2>
  constexpr tagged_pair<Tag1(V1), Tag2(V2)> make_tagged_pair(T1&& x, T2&& y);
}}}

namespace std {
  // \ref{tagged.astuple}, tuple-like access to tagged
  template <class Base, class... Tags>
  struct tuple_size<experimental::ranges_v1::tagged<Base, Tags...>>;
  
  template <size_t N, class Base, class... Tags>
  struct tuple_element<N, experimental::ranges_v1::tagged<Base, Tags...>>;
}
\end{codeblock}
}

{\color{addclr}
\pnum
Any entities declared or defined directly in namespace \tcode{std} in header \tcode{<utility>}
that are not already defined in namespace \tcode{std::experimental::ranges_v1} in header
\tcode{<experimental/ranges_v1/utility>} are imported with
\grammarterm{using-declaration}{s}~(\cxxref{namespace.udecl}). \enterexample
\begin{codeblock}
namespace std { namespace experimental { namespace ranges_v1 {
  using std::pair;
  using std::make_pair;
  // ... others
}}}
\end{codeblock}
\exitexample

\pnum
Any nested namespaces defined directly in namespace \tcode{std} in header \tcode{<utility>}
that are not already defined in namespace \tcode{std::experimental::ranges_v1} in header
\tcode{<experimental/ranges_v1/utility>} are aliased with a
\grammarterm{namespace-alias-definition}~(\cxxref{namespace.alias}). \enterexample
\begin{codeblock}
namespace std { namespace experimental { namespace ranges_v1 {
  namespace rel_ops = std::rel_ops;
}}}
\end{codeblock}
\exitexample
}

\setcounter{subsection}{1}
\rSec2[utility.swap]{swap}

\ednote{The following edits to the two \tcode{swap} overloads are to be applied to the versions in
namespace \tcode{std}.}

\indexlibrary{\idxcode{swap}}%
\begin{itemdecl}
template<class T>
  @\added{requires experimental::ranges_v1::Movable<T>()}@
void swap(T& a, T& b) noexcept(@\seebelow@);
\end{itemdecl}

\begin{itemdescr}
\pnum
\remark The expression inside \tcode{noexcept} is equivalent to:

\begin{codeblock}
is_nothrow_move_constructible<T>::value &&
is_nothrow_move_assignable<T>::value @\added{\&\&}@
@\added{is_nothrow_destructible<T>::value}@
\end{codeblock}

{\color{addclr}
\pnum
\remark
A library implementor is free to omit the \tcode{requires} clause so long as
this function does not participate in overload resolution if the following
is false
\begin{codeblock}
is_move_constructible<T>::value &&
is_move_assignable<T>::value &&
is_destructible<T>::value
\end{codeblock}
\ednote{This is to support compilation environments that do not support concepts.}
}

\removed{
\pnum
\requires
Type
\tcode{T}
shall be
\tcode{MoveConstructible} (Table~\cxxref{moveconstructible})
and
\tcode{MoveAssignable} (Table~\cxxref{moveassignable}).
}

\pnum
\effects
Exchanges values stored in two locations.
\end{itemdescr}

\indexlibrary{\idxcode{swap}}%
\begin{itemdecl}
template <class T, size_t N>
  @\added{requires requires (remove_all_extents_t<T>\& t) \{ swap(t, t); \}}@
void swap(T (&a)[N], T (&b)[N]) noexcept(noexcept(swap(*a, *b)));}
\end{itemdecl}

\begin{itemdescr}
\pnum
\requires
\tcode{a[i]} shall be swappable with~(\ref{concepts.lib.corelang.swappable}) \tcode{b[i]}
for all \tcode{i} in the range \range{0}{N}.

{\color{addclr}
\pnum
\remark
A library implementor is free to omit the \tcode{requires} clause so long as
this function does not participate in overload resolution if the following
expression is ill-formed
\begin{codeblock}
swap(declval<remove_all_extents_t<T>&>(), declval<remove_all_extents_t<T>&>())
\end{codeblock}
\ednote{This is to support compilation environments that do not support concepts.}
}

\pnum
\effects \tcode{swap_ranges(a, a + N, b)}
\end{itemdescr}

\rSec2[utility.exchange]{exchange}

\begin{itemdecl}
template <@\removed{class}\added{Movable}@ T, class U=T>
  @\added{requires Assignable<T\&, U>()}@
T exchange(T& obj, U&& new_val);
\end{itemdecl}

\begin{itemdescr}
\pnum
\effects
Equivalent to:

\begin{codeblock}
T old_val = std::move(obj);
obj = std::forward<U>(new_val);
return old_val;
\end{codeblock}
\end{itemdescr}

\setcounter{section}{8}
\rSec1[function.objects]{Function objects}

\setcounter{Paras}{1}
\pnum
\synopsis{Header \tcode{<\added{experimental/ranges_v1/}functional>} synopsis}

\begin{codeblock}
@\added{namespace std \{ namespace experimental \{ namespace ranges_v1 \{}@
  template <class T = void>
    @\added{requires EqualityComparable<T>() || Same<T, void>()}@
  struct equal_to;
  
  template <class T = void>
    @\added{requires EqualityComparable<T>() || Same<T, void>()}@
  struct not_equal_to;
  
  template <class T = void>
    @\added{requires TotallyOrdered<T>() || Same<T, void>()}@
  struct greater;
  
  template <class T = void>
    @\added{requires TotallyOrdered<T>() || Same<T, void>()}@
  struct less;
  
  template <class T = void>
    @\added{requires TotallyOrdered<T>() || Same<T, void>()}@
  struct greater_equal;
  
  template <class T = void>
    @\added{requires TotallyOrdered<T>() || Same<T, void>()}@
  struct less_equal;
  
  @\added{struct identity;}@
@\added{\}\}\}}@
\end{codeblock}

{\color{addclr}
\pnum
Any entities declared or defined directly in namespace \tcode{std} in header \tcode{<functional>}
that are not already defined in namespace \tcode{std::experimental::ranges_v1} in header
\tcode{<experimental/ranges_v1/functional>} are imported with
\grammarterm{using-declaration}{s}~(\cxxref{namespace.udecl}). \enterexample
\begin{codeblock}
namespace std { namespace experimental { namespace ranges_v1 {
  using std::reference_wrapper;
  using std::ref;
  // ... others
}}}
\end{codeblock}
\exitexample

\pnum
Any nested namespaces defined directly in namespace \tcode{std} in header \tcode{<functional>}
that are not already defined in namespace \tcode{std::experimental::ranges_v1} in header
\tcode{<experimental/ranges_v1/functional>} are aliased with a
\grammarterm{namespace-alias-definition}~(\cxxref{namespace.alias}). \enterexample
\begin{codeblock}
namespace std { namespace experimental { namespace ranges_v1 {
  namespace placeholders = std::placeholders;
}}}
\end{codeblock}
\exitexample
}

\setcounter{subsection}{5}
\rSec2[comparisons]{Comparisons}

\pnum
The library provides basic function object classes for all of the comparison
operators in the language~(\cxxref{expr.rel}, \cxxref{expr.eq}).

\indexlibrary{\idxcode{equal_to}}%
\begin{itemdecl}
template <class T = void>
  @\added{requires EqualityComparable<T>() || Same<T, void>()}@
struct equal_to {
  constexpr bool operator()(const T& x, const T& y) const;
  typedef T first_argument_type;
  typedef T second_argument_type;
  typedef bool result_type;
};
\end{itemdecl}

\begin{itemdescr}
\pnum
\tcode{operator()} returns \tcode{x == y}.
\end{itemdescr}

\indexlibrary{\idxcode{not_equal_to}}%
\begin{itemdecl}
template <class T = void>
  @\added{requires EqualityComparable<T>() || Same<T, void>()}@
struct not_equal_to {
  constexpr bool operator()(const T& x, const T& y) const;
  typedef T first_argument_type;
  typedef T second_argument_type;
  typedef bool result_type;
};
\end{itemdecl}

\begin{itemdescr}
\pnum
\tcode{operator()} returns \tcode{x != y}.
\end{itemdescr}

\indexlibrary{\idxcode{greater}}%
\begin{itemdecl}
template <class T = void>
  @\added{requires TotallyOrdered<T>() || Same<T, void>()}@
struct greater {
  constexpr bool operator()(const T& x, const T& y) const;
  typedef T first_argument_type;
  typedef T second_argument_type;
  typedef bool result_type;
};
\end{itemdecl}

\begin{itemdescr}
\pnum
\tcode{operator()} returns \tcode{x > y}.
\end{itemdescr}

\indexlibrary{\idxcode{less}}%
\begin{itemdecl}
template <class T = void>
  @\added{requires TotallyOrdered<T>() || Same<T, void>()}@
struct less {
  constexpr bool operator()(const T& x, const T& y) const;
  typedef T first_argument_type;
  typedef T second_argument_type;
  typedef bool result_type;
};
\end{itemdecl}

\begin{itemdescr}
\pnum
\tcode{operator()} returns \tcode{x < y}.
\end{itemdescr}

\indexlibrary{\idxcode{greater_equal}}%
\begin{itemdecl}
template <class T = void>
  @\added{requires TotallyOrdered<T>() || Same<T, void>()}@
struct greater_equal {
  constexpr bool operator()(const T& x, const T& y) const;
  typedef T first_argument_type;
  typedef T second_argument_type;
  typedef bool result_type;
};
\end{itemdecl}

\begin{itemdescr}
\pnum
\tcode{operator()} returns \tcode{x >= y}.
\end{itemdescr}

\indexlibrary{\idxcode{less_equal}}%
\begin{itemdecl}
template <class T = void>
  @\added{requires TotallyOrdered<T>() || Same<T, void>()}@
struct less_equal {
  constexpr bool operator()(const T& x, const T& y) const;
  typedef T first_argument_type;
  typedef T second_argument_type;
  typedef bool result_type;
};
\end{itemdecl}

\begin{itemdescr}
\pnum
\tcode{operator()} returns \tcode{x <= y}.
\end{itemdescr}

\indexlibrary{\idxcode{equal_to<>}}%
\begin{itemdecl}
template <> struct equal_to<void> {
  template <class T, class U>
    @\added{requires EqualityComparable<T, U>()}@
  constexpr auto operator()(T&& t, U&& u) const
    -> decltype(std::forward<T>(t) == std::forward<U>(u));

  typedef @\unspec@ is_transparent;
};
\end{itemdecl}

\begin{itemdescr}
\pnum
\tcode{operator()} returns \tcode{std::forward<T>(t) == std::forward<U>(u)}.
\end{itemdescr}

\indexlibrary{\idxcode{not_equal_to<>}}%
\begin{itemdecl}
template <> struct not_equal_to<void> {
  template <class T, class U>
    @\added{requires EqualityComparable<T, U>()}@
  constexpr auto operator()(T&& t, U&& u) const
    -> decltype(std::forward<T>(t) != std::forward<U>(u));

  typedef @\unspec@ is_transparent;
};
\end{itemdecl}

\begin{itemdescr}
\pnum
\tcode{operator()} returns \tcode{std::forward<T>(t) != std::forward<U>(u)}.
\end{itemdescr}

\indexlibrary{\idxcode{greater<>}}%
\begin{itemdecl}
template <> struct greater<void> {
  template <class T, class U>
    @\added{requires TotallyOrdered<T, U>()}@
  constexpr auto operator()(T&& t, U&& u) const
    -> decltype(std::forward<T>(t) > std::forward<U>(u));

  typedef @\unspec@ is_transparent;
};
\end{itemdecl}

\begin{itemdescr}
\pnum
\tcode{operator()} returns \tcode{std::forward<T>(t) > std::forward<U>(u)}.
\end{itemdescr}

\indexlibrary{\idxcode{less<>}}%
\begin{itemdecl}
template <> struct less<void> {
  template <class T, class U>
    @\added{requires TotallyOrdered<T, U>()}@
  constexpr auto operator()(T&& t, U&& u) const
    -> decltype(std::forward<T>(t) < std::forward<U>(u));

  typedef @\unspec@ is_transparent;
};
\end{itemdecl}

\begin{itemdescr}
\pnum
\tcode{operator()} returns \tcode{std::forward<T>(t) < std::forward<U>(u)}.
\end{itemdescr}

\indexlibrary{\idxcode{greater_equal<>}}%
\begin{itemdecl}
template <> struct greater_equal<void> {
  template <class T, class U>
    @\added{requires TotallyOrdered<T, U>()}@
  constexpr auto operator()(T&& t, U&& u) const
    -> decltype(std::forward<T>(t) >= std::forward<U>(u));

  typedef @\unspec@ is_transparent;
};
\end{itemdecl}

\begin{itemdescr}
\pnum
\tcode{operator()} returns \tcode{std::forward<T>(t) >= std::forward<U>(u)}.
\end{itemdescr}

\indexlibrary{\idxcode{less_equal<>}}%
\begin{itemdecl}
template <> struct less_equal<void> {
  template <class T, class U>
    @\added{requires TotallyOrdered<T, U>()}@
  constexpr auto operator()(T&& t, U&& u) const
    -> decltype(std::forward<T>(t) <= std::forward<U>(u));

  typedef @\unspec@ is_transparent;
};
\end{itemdecl}

\begin{itemdescr}
\pnum
\tcode{operator()} returns \tcode{std::forward<T>(t) <= std::forward<U>(u)}.
\end{itemdescr}

\pnum
For templates \tcode{greater}, \tcode{less}, \tcode{greater_equal}, and
\tcode{less_equal}, the specializations for any pointer type yield a total order,
even if the built-in operators \tcode{<}, \tcode{>}, \tcode{<=}, \tcode{>=}
do not.

\ednote{After subsection 20.9.12 [unord.hash] add the following subsection:}

\setcounter{subsection}{12}
\begin{addedblock}
\rSec2[func.identity]{Class identity}

\indexlibrary{\idxcode{identity}}%
\begin{itemdecl}
struct identity {
  template <class T>
  constexpr T&& operator()(T&& t) const noexcept;
};
\end{itemdecl}

\begin{itemdescr}
\pnum
\tcode{operator()} returns \tcode{std::forward<T>(t)}.

\ednote{REVIEW: From Stephan T. Lavavej: "[This] \tcode{identity} functor, being
a non-template, clashes with any attempt to provide \tcode{identity<T>::type}." <Insert
bikeshed naming discussion here>.}
\end{itemdescr}
\end{addedblock}


\setcounter{section}{14}

{\color{addclr}
\rSec1[taggedtup]{Tagged tuple-like types}

\rSec2[taggedtup.general]{In general}

\pnum The library provides a template for augmenting a tuple-like type with named element accessor
member functions. The library also provides several templates that provide access to \tcode{tagged}
objects as if they were \tcode{tuple} objects (see~\cxxref{tuple.elem}).

\ednote{This type exists so that the algorithms can return pair- and tuple-like objects with named
accessors, as requested by LEWG. Rather than create a bunch of one-off class types with no relation
to pair and tuple, I opted instead to create a general utility. I'll note that this functionality
can be merged into \tcode{pair} and \tcode{tuple} directly with minimal breakage, but I opted for
now to keep it separate.}

\rSec2[taggedtup.tagged]{Class template \tcode{tagged}}

\pnum
Class template \tcode{tagged} augments a tuple-like class type by giving it named accessors. It is
used to define the alias templates \tcode{tagged_pair}~(\ref{tagged.pairs}) and
\tcode{tagged_tuple}~(\ref{tagged.tuple}).

\pnum In the class synopsis below, let $i$ be in the range
\range{0}{sizeof...(Tags)} in order and $T_i$ be the $i^{th}$ type in \tcode{Tags}, where indexing
is zero-based.

\indexlibrary{\idxcode{tagged}}%
\begin{codeblock}
// defined in header <experimental/ranges_v1/utility>

namespace std { namespace experimental { namespace ranges_v1 {
  template <class Base, class... Tags>
  struct tagged :
    Base, @\textit{TAGGET}@(tagged<Base, Tags...>, @$T_i$@, @$i$@)... { // \seebelow
    using Base::Base;
    tagged() = default;
    tagged(tagged&&) = default;
    tagged(const tagged&) = default;
    tagged &operator=(tagged&&) = default;
    tagged &operator=(const tagged&) = default;
    template <typename Other>
      requires Constructible<Base, Other>()
    tagged(tagged<Other, Tags...> &&that) noexcept(@\seebelow@);
    template <typename Other>
      requires Constructible<Base, const Other&>()
    tagged(tagged<Other, Tags...> const &that);
    template <typename Other>
      requires Assignable<Base&, Other>()
    tagged& operator=(tagged<Other, Tags...>&& that) noexcept(@\seebelow@);
    template <typename Other>
      requires Assignable<Base&, const Other&>()
    tagged& operator=(const tagged<Other, Tags...>& that);
    template <class U>
      requires Assignable<Base&, U>() && !Same<decay_t<U>, tagged>()
    tagged& operator=(U&& u) noexcept(@\seebelow@);
    void swap(tagged& that) noexcept(@\seebelow@)
      requires Swappable<Base&>();
  };

  template <class Base, class... Tags>
    requires Swappable<Base&>()
  void swap(tagged<Base, Tags...>& x, tagged<Base, Tags...>& y) noexcept(@\seebelow@);
}}}
\end{codeblock}

\pnum The size of the \tcode{Tags} parameter pack shall be less than or equal to
\tcode{tuple_size<Base>::value}.

\pnum A \techterm{tagged getter} is an empty trivial class type that has a named member function that
returns a reference to a member of a tuple-like object that is assumed to be derived from the getter
class. The tuple-like type of a tagged getter is called its \techterm{DerivedCharacteristic}.
The index of the tuple element returned from the getter's member functions is called its
\techterm{ElementIndex}. The name of the getter's member function is called its
\techterm{ElementName}

\pnum A tagged getter class with DerivedCharacteristic \tcode{\textit{D}}, ElementIndex
\tcode{\textit{N}}, and ElementName \tcode{\textit{name}} shall provide the following interface:

\begin{codeblock}
struct @\xname{\textit{TAGGED_GETTER}}@ {
  constexpr decltype(auto) @$name$@() &       { return get<@$N$@>(static_cast<@$D$@&>(*this)); }
  constexpr decltype(auto) @$name$@() &&      { return get<@$N$@>(static_cast<@$D$@&&>(*this)); }
  constexpr decltype(auto) @$name$@() const & { return get<@$N$@>(static_cast<const @$D$@&>(*this)); }
};
\end{codeblock}

\pnum A \techterm{tag specifier} is a type that facilitates a mapping from a tuple-like type and an
element index into a \textit{tagged getter} that gives named access to the element at that index.
The types in the \tcode{Tags} parameter pack shall be tag specifiers. The tag specifiers in the
\tcode{Tags} parameter pack shall be unique. \enternote The mapping mechanism from tag specifier to
tagged getter is unspecified.\exitnote

\pnum Let \tcode{\textit{TAGGET}(D, T, $N$)} name a tagged getter type that gives named
access to the $N$-th element of the tuple-like type \tcode{D}.

\pnum It shall not be possible to delete an instance of class template \tcode{tagged} through a
pointer to any base other than \tcode{Base}.

\pnum A \techterm{tagged type} is a unary function type whose return type is a tag specifier. Let
\tcode{\textit{TAGSPEC}(F)} name the tag specifier of the tagged type \tcode{F}, and let
\tcode{\textit{TAGELEM}(F)} name the argument type of the tagged type \tcode{F}.

\indexlibrary{\idxcode{tagged}!\idxcode{tagged}}
\begin{itemdecl}
template <typename Other>
  requires Constructible<Base, Other>()
tagged(tagged<Other, Tags...> &&that) noexcept(@\seebelow@);
\end{itemdecl}

\begin{itemdescr}
\pnum
\remarks The expression in the \tcode{noexcept} is equivalent to:

\begin{codeblock}
is_nothrow_constructible<Base, Other>::value
\end{codeblock}

\pnum
\effects Initializes \tcode{Base} with \tcode{static_cast<Other\&\&>(that)}.
\end{itemdescr}

\indexlibrary{\idxcode{tagged}!\idxcode{tagged}}
\begin{itemdecl}
template <typename Other>
  requires Constructible<Base, const Other&>()
tagged(const tagged<Other, Tags...>& that);
\end{itemdecl}

\begin{itemdescr}
\pnum
\effects Initializes \tcode{Base} with \tcode{static_cast<const Other\&>(that)}.
\end{itemdescr}

\indexlibrary{\idxcode{operator=}!\idxcode{tagged}}
\indexlibrary{\idxcode{tagged}!\idxcode{operator=}}
\begin{itemdecl}
template <typename Other>
  requires Assignable<Base&, Other>()
tagged& operator=(tagged<Other, Tags...>&& that) noexcept(@\seebelow@);
\end{itemdecl}

\begin{itemdescr}
\pnum
\remarks The expression in the \tcode{noexcept} is equivalent to:

\begin{codeblock}
is_nothrow_assignable<Base&, Other>::value
\end{codeblock}

\pnum
\effects Assigns \tcode{static_cast<Other\&\&>(that)} to \tcode{static_cast<Base\&>(*this)}.

\pnum
\returns \tcode{*this}.
\end{itemdescr}

\indexlibrary{\idxcode{operator=}!\idxcode{tagged}}
\indexlibrary{\idxcode{tagged}!\idxcode{operator=}}
\begin{itemdecl}
template <typename Other>
  requires Assignable<Base&, const Other&>()
tagged& operator=(const tagged<Other, Tags...>& that);
\end{itemdecl}

\begin{itemdescr}
\pnum
\effects Assigns \tcode{static_cast<const Other\&>(that)} to \tcode{static_cast<Base\&>(*this)}.

\pnum
\returns \tcode{*this}.
\end{itemdescr}

\indexlibrary{\idxcode{operator=}!\idxcode{tagged}}
\indexlibrary{\idxcode{tagged}!\idxcode{operator=}}
\begin{itemdecl}
template <class U>
  requires Assignable<Base&, U>() && !Same<decay_t<U>, tagged>()
tagged& operator=(U&& u) noexcept(@\seebelow@);
\end{itemdecl}

\begin{itemdescr}
\pnum
\remarks The expression in the \tcode{noexcept} is equivalent to:

\begin{codeblock}
is_nothrow_assignable<Base&, U>::value
\end{codeblock}

\pnum
\effects Assigns \tcode{std::forward<U>(u)} to \tcode{static_cast<Base\&>(*this)}.

\pnum
\returns \tcode{*this}.
\end{itemdescr}

\indexlibrary{\idxcode{swap}!\idxcode{tagged}}
\indexlibrary{\idxcode{tagged}!\idxcode{swap}}
\begin{itemdecl}
void swap(tagged& rhs) noexcept(@\seebelow@)
  requires Swappable<Base&>();
\end{itemdecl}

\begin{itemdescr}
\pnum
\remarks The expression in the \tcode{noexcept} is equivalent to:

\begin{codeblock}
noexcept(swap(declval<Base&>(), declval<Base&>()))
\end{codeblock}

\pnum
\effects Calls \tcode{swap} on the result of applying \tcode{static_cast} to \tcode{*this} and
\tcode{that}.

\pnum
\throws Nothing unless the call to \tcode{swap} on the \tcode{Base} sub-objects throws.

\end{itemdescr}

\rSec2[tagged.special]{\tcode{tagged} specialized algorithms}

\indexlibrary{\idxcode{swap}!\tcode{tagged}}%

\begin{itemdecl}
template<class Base, class... Tags>
  requires Swappable<Base&>()
void swap(tagged<Base, Tags...>& x, tagged<Base, Tags...>& y)
  noexcept(noexcept(x.swap(y)));
\end{itemdecl}

\begin{itemdescr}
\pnum
\effects \tcode{x.swap(y)}
\end{itemdescr}

\rSec2[tagged.astuple]{Tuple-like access to \tcode{tagged}}

\indexlibrary{\idxcode{tuple_size}}%
\indexlibrary{\idxcode{tuple_element}}%
\begin{itemdecl}
namespace std {
  template <class Base, class... Tags>
  struct tuple_size<experimental::ranges_v1::tagged<Base, Tags...>>
    : tuple_size<Base> { };

  template <size_t N, class Base, class... Tags>
  struct tuple_element<N, experimental::ranges_v1::tagged<Base, Tags...>>
    : tuple_element<N, Base> { };
}
\end{itemdecl}

\rSec2[tagged.pairs]{Alias template \tcode{tagged_pair}}

\begin{codeblock}
// defined in header <experimental/ranges_v1/utility>

namespace std { namespace experimental { namespace ranges_v1 {
  // ...
  template <class T1, class T2>
  using tagged_pair = tagged<pair<@\textit{TAGELEM}@(T1), @\textit{TAGELEM}@(T2)>,
                             @\textit{TAGSPEC}@(T1), @\textit{TAGSPEC}@(T2)>;
}}}
\end{codeblock}

\pnum Types \tcode{T1}  and \tcode{T2} shall be tagged types~(\ref{taggedtup.tagged}).

\pnum \enterexample
\begin{codeblock}
// See \ref{alg.tagspec}:
tagged_pair<tag::min(int), tag::max(int)> p{0, 1};
assert(&p.min() == &p.first);
assert(&p.max() == &p.second);
\end{codeblock}
\exitexample

\rSec3[tagged.pairs.creation]{Tagged pair creation functions}

\indexlibrary{\idxcode{make_tagged_pair}}%
\begin{itemdecl}
// defined in header <experimental/ranges_v1/utility>

namespace std { namespace experimental { namespace ranges_v1 {
  template <class Tag1, class Tag2, class T1, class T2>
    constexpr tagged_pair<Tag1(V1), Tag2(V2)> make_tagged_pair(T1&& x, T2&& y);
}}}
\end{itemdecl}

\begin{itemdescr}
\pnum
\returns \tcode{tagged_pair<Tag1(V1), Tag2(V2)>(std::forward<T1>(x), std::forward<T2>(y))};

where \tcode{V1} and \tcode{V2} are determined as follows: Let \tcode{Ui} be
\tcode{decay_t<Ti>} for each \tcode{Ti}. Then each \tcode{Vi} is \tcode{X\&}
if \tcode{Ui} equals \tcode{reference_wrapper<X>}, otherwise \tcode{Vi} is
\tcode{Ui}.

\pnum
\enterexample
In place of:

\begin{codeblock}
  return tagged_pair<tag::min(int), tag::max(double)>(5, 3.1415926);   // explicit types
\end{codeblock}

a \Cpp program may contain:

\begin{codeblock}
  return make_tagged_pair<tag::min, tag::max>(5, 3.1415926);           // types are deduced
\end{codeblock}
\exitexample
\end{itemdescr}

{\color{addclr}
\rSec2[tagged.tuple]{Alias template \tcode{tagged_tuple}}

\synopsis{Header \tcode{<experimental/ranges_v1/tuple>} synopsis}

\begin{codeblock}
namespace std { namespace experimental { namespace ranges_v1 {
  template <class... Types>
  using tagged_tuple = tagged<tuple<@\textit{TAGELEM}@(Types)...>,
                              @\textit{TAGSPEC}@(Types)...>;

  template<class...Tags, class... Types>
    constexpr tagged_tuple<Tags(@\textit{VTypes}@)...>> make_tagged_tuple(Types&&... t);
}}}
\end{codeblock}

\pnum
Any entities declared or defined in namespace \tcode{std} in header \tcode{<tuple>}
that are not already defined in namespace \tcode{std::experimental::ranges_v1} in header
\tcode{<experimental/ranges_v1/tuple>} are imported with
\grammarterm{using-declaration}{s}~(\cxxref{namespace.udecl}). \enterexample
\begin{codeblock}
namespace std { namespace experimental { namespace ranges_v1 {
  using std::tuple;
  using std::make_tuple;
  // ... others
}}}
\end{codeblock}
\exitexample
}

\begin{codeblock}
template <class... Types>
using tagged_tuple = tagged<tuple<@\textit{TAGELEM}@(Types)...>,
                            @\textit{TAGSPEC}@(Types)...>;
\end{codeblock}

\pnum The types in parameter pack \tcode{Types} shall be tagged types~(\ref{taggedtup.tagged}).

\pnum \enterexample
\begin{codeblock}
// See \ref{alg.tagspec}:
tagged_tuple<tag::in(char*), tag::out(char*)> t{0, 0};
assert(&t.in() == &get<0>(t));
assert(&t.out() == &get<1>(t));
\end{codeblock}
\exitexample

\rSec3[tagged.tuple.creation]{Tagged tuple creation functions}

\pnum
In the function descriptions that follow, let $i$ be in the range \range{0}{sizeof...(Types)}
in order and let $T_i$ be the $i^{th}$ type in a template parameter pack named \tcode{Types},
where indexing is zero-based.

\indexlibrary{\idxcode{make_tagged_tuple}}%
\indexlibrary{\idxcode{tagged_tuple}!\idxcode{make_tagged_tuple}}%
\begin{itemdecl}
template<class...Tags, class... Types>
  constexpr tagged_tuple<Tags(@\textit{VTypes}@)...>> make_tagged_tuple(Types&&... t);
\end{itemdecl}

\begin{itemdescr} \pnum Let \tcode{$U_i$} be \tcode{decay_t<$T_i$>} for each
$T_i$ in \tcode{Types}. Then each $V_i$ in \tcode{VTypes} is
\tcode{X\&} if $U_i$ equals \tcode{reference_wrapper<X>}, otherwise
$V_i$ is $U_i$.

\pnum
\returns \tcode{tagged_tuple<Tags(VTypes)...>(std::forward<Types>(t)...)}.

\pnum
\enterexample

\begin{codeblock}
int i; float j;
make_tagged_tuple<tag::in1, tag::in2, tag::out>(1, ref(i), cref(j))
\end{codeblock}

creates a tagged tuple of type

\begin{codeblock}
tagged_tuple<tag::in1(int), tag::in2(int&), tag::out(const float&)>
\end{codeblock}
\exitexample
\end{itemdescr}
}
