%!TEX root = stl2-ts.tex
\infannex{sentrework}{Iterator Design Rationale}

\pnum
The design of the iterator and sentinel concepts in this document differs
substantially from the design presented in N4560. In short, N4560 had separate
``weak'' and ``strong'' variants of \tcode{Iterator}, \tcode{InputIterator}, and
\tcode{OutputIterator}; in the current design the ``strong''
(\tcode{EqualityComparable}~(\ref{concepts.lib.compare.equalitycomparable})) variants have been removed and the \tcode{Weak}
prefix has been elided from the names of the ``weak'' variants. This rationale
justifies the change and clarifies that it does not diminish the expressive
power of iterator / sentinel pairs and ranges.

\rSec1[sentrework.problem]{Problem statement}

\pnum
Consider a threshold sentinel \tcode{S} with state consisting of a single
integer \tcode{bound}. When compared for equality with an iterator, this
sentinel's \tcode{==} returns \tcode{true} if and only if the iterator denotes
a value greater than or equal to \tcode{bound}. \tcode{S} seems like a useful
tool, and it's simple enough to be described completely in a couple of
sentences. One would expect it to be expressible within the sentinel model of
N4560. Given an array of integers:
\begin{codeblock}
  int i[] = {2,1,3};
\end{codeblock}
By any sensible definition of ``the same range,'' \range{i+1}{S\{2\}} and
\range{i+1}{S\{3\}} (where \tcode{S\{n\}} is a threshold sentinel with
\tcode{bound} of \tcode{n}) denote the same range. \tcode{*(i+1) == 1} is less
than both bounds (\tcode{2} and \tcode{3}), and \tcode{*(i+2) == 3} is greater
than or equal to both bounds, so both ranges have length of one and denote only
the single integer with value \tcode{1}.

\pnum
N4560 requires iterators and their sentinels to satisfy the cross-type
\tcode{EqualityComparable<T,U>()} concept~(\ref{concepts.lib.compare.equalitycomparable}).
\tcode{EqualityComparable} effectively requires the values of the two
participant types \tcode{T} and \tcode{U} to be put into correspondence: if two
objects \tcode{t1} and \tcode{t2} of one type both compare equal to some value
of the other type, those objects must be equal. The meaning of ``be equal'' is
very clearly defined in the Ranges TS by the semantics of equality preserving
expressions~(\ref{concepts.lib.general.equality}). If \tcode{t1} and \tcode{t2}
are equal, the result of evaluating any expression required to be equality
preserving must be the same for both \tcode{t1} and \tcode{t2}. Informally,
the correspondence enforced by \tcode{EqualityComparable} requires values of the
participant types to both represent the same abstract entities as personified by
their common type.

\pnum
Returning to the example above, the consequences of the
\tcode{EqualityComparable} requirement are clear: since \tcode{i+2 == S\{2\}}
and \tcode{i+2 == S\{3\}}, \tcode{S\{2\}} and \tcode{S\{3\}} must be equal. If
they are equal, then the ranges \range{i+0}{S\{2\}} and \range{i+0}{S\{3\}} must
also denote the same range due to the equality preservation of the expressions
involved. This is, however, not the case since \tcode{*(i+0) == 2} is greater
than or equal to \tcode{2} but not greater than or equal to \tcode{3}.
\tcode{EqualityComparable<S,int*>()} and therefore \tcode{Sentinel<S,int*>()}
are not satisfied; \tcode{S} is not a valid sentinel for \tcode{int*}.

\pnum
Clearly there is a large class of useful sentinels that the N4560 model does
not admit: stateful sentinels which might compare equal to the same iterator
when they have differing state. N4560 constrains iterators and sentinels to
both represent the abstract notion of ``position,'' which is not a good fit
for stateful sentinels.

\rSec1[sentrework.approach]{Fixing the problem}

\pnum
It seems odd to claim that the sentinel semantics are broken when there are two
working implementations of the TS~(\cite{range-v3}, \cite{cmcstl2}) that haven't
had any issues due to this semantic inconsistency. The clear implication is that
the semantics of the N4560 model are not an exact match for the requirements of
the algorithms. The semantics are likely overconstrained and could be relaxed
without losing the properties the algorithms need to support equational
reasoning.

\rSec2[sentrework.approach.alg]{Look at the Algorithms}

\pnum
As is often the case, a problem in the semantics is best fixed by examining
the algorithms and determining what behaviors they require to ensure their
postconditions. The properties the algorithms require for sentinels are (where
\tcode{s} denotes a sentinel value and \tcode{i} an iterator value):
\begin{itemize}
\item \tcode{i == s}, \tcode{s == i}, \tcode{i != s}, and \tcode{s != i} are
      equality-preserving expressions with the same domain.
\item Symmetry: \tcode{bool(i == s) == bool(s == i)} and
      \tcode{bool(i != s) == bool(s != i)}.
\item Complement: \tcode{bool(i != s) == !bool(i == s)}.
\item \tcode{i == s} is well-defined when \range{i}{s} denotes a range.
\end{itemize}

\pnum
The first three properties aren't specific to iterators and sentinels; they are
general characteristics of the operators \tcode{==} and \tcode{!=} that are
intuitively expected to hold whenever they are defined. As such, the
\tcode{EqualityComparable} concepts should require them to hold and this
particular set of properties can then be presented as a relaxation of
\tcode{EqualityComparable<T,U>()}. The current design does exactly that,
defining the concept
\tcode{Weakly\-Equal\-ity\-Comp\-arable}~(\ref{concepts.lib.compare.equalitycomparable})
with exactly the specified semantic requirements, and reformulating both the
single and cross-type variants of \tcode{EqualityComparable} as refinements
thereof.

\pnum
Notably, nothing in this list of properties requires comparison of sentinels
with other sentinels -- it's not a useful feature for generic code. The
requirement for equality comparison of sentinels in N4560 is a checkbox feature
to enable satisfaction of cross-type \tcode{EqualityComparable}. Absent the
\tcode{EqualityComparable} requirement, nothing in the design requires equality
comparison for sentinels. It's nonsensical to require sentinels to have
\tcode{Regular} types if \tcode{==} is never required to be valid or useful;
they should be \tcode{Semiregular}.

\pnum
The list of properties also does not require that iterator types be
\tcode{EqualityComparable} to participate in the \tcode{Sentinel} relationship.
Just as removing the cross-type \tcode{EqualityComparable} requirement
makes it unnecessary for sentinels to be \tcode{EqualityComparable} with
other sentinels, it also removes the necessity for the iterator type to be
\tcode{EqualityComparable}. In N4560 terms, \tcode{Sentinel} should be a
relationship between a \tcode{WeakIterator} type that denotes an element of a
range and a \tcode{Semiregular} type that represents an end-of-range calculation
as a predicate over \tcode{WeakIterator} values embodied in the form of the
\tcode{==} operator, i.e.:
\begin{codeblock}
template <class S, class I>
concept bool Sentinel() {
  return Semiregular<S>() && WeakIterator<I>() &&
    WeaklyEqualityComparable<S, I>();
}
\end{codeblock}
a non-weak iterator is capable of both denoting an element and the end of a
range. In other words, a ``strong'' iterator is a type that is simultaneously a
weak iterator and a sentinel for that weak iterator:
\begin{codeblock}
template <class I>
concept bool Iterator() {
  return Sentinel<I, I>() &&
    EqualityComparable<I>(); // Keep the semantic property that == means
                             // substitutable in equality-preserving expressions
}
\end{codeblock}

\rSec1[sentrework.ram]{Ramifications}

\pnum
Correcting the inconsistency in the \tcode{Sentinel} semantics creates a class
of ``weak'' ranges -- ranges with \tcode{WeakInput} or \tcode{WeakOutput}
iterator types -- that the algorithms do not admit. It would appear that
algorithms that accept input (output) ranges should possibly be relaxed to
accept weak input (weak output) ranges instead. How would doing so constrain
the implementations of the algorithms? Do weak ranges have less expressive power?

\rSec2[sentrework.ram.singlepass]{Single-pass Algorithms}

\pnum
Algorithms that accept input or output iterators necessarily perform at most one
pass over the range of elements those iterators reference. N4560's weak input and
weak output iterators are consistent with those single-pass semantics. The
sole distinction is support for equality comparison. How do algorithms use
equality comparison with iterators for single-pass ranges?

\pnum
Oddly enough, the answer to the question posed is that algorithms need not ever
use equality comparison on single-pass iterators. In the iterator pair model its
necessary to compare an iterator denoting the current position in the range with
the supplied end iterator to determine when the algorithm reaches the end of the
range. An algorithm \textit{could} compare the ``current'' iterator with itself
or a copy thereof, but it would be pointless to do so. An algorithm
\textit{cannot} compare the ``current'' iterator with a copy of an earlier value
it held because that operation is not required to be valid for single-pass
iterators.

\pnum
The case is similar for the iterator / sentinel model. An algorithm compares the
iterator with the sentinel to determine when the end of the range is reached,
but cannot compare the iterator with a copy of an older value. It would again be
pointless to compare the iterator with itself or a copy; there's no reason to
test things known to be equal for equality. Consequently, relaxing the
algorithms that use single-pass ranges to ``weak'' single-pass ranges doesn't
reduce their power.

\rSec2[sentrework.ram.merge]{Why ``strong'' Iterators?}

\pnum
Relaxing the algorithms necessitates relaxing the iterator operations
(\tcode{advance} et al.~(\ref{iterator.operations})), which the algorithms use.
Indeed there are many components in the specification that require a ``strong''
iterator not because they need to compare iterator values for equality, but
because iterator / sentinel pairs in N4560 necessarily used strong iterators.
Once all such occurrences are relaxed, it becomes apparent that there is no need
for separate ``weak'' and ``strong'' concepts for input and output iterators;
the strong concepts are unused. The current design removes N4560's
\tcode{InputIterator} and \tcode{OutputIterator} concepts and strips the
\tcode{Weak} prefix from the names of N4560's \tcode{WeakInputIterator} and
\tcode{WeakOutputIterator} since the name distinction is no longer necessary.

\pnum
After removing the distinction between weak and strong input and output
iterators, the \tcode{Iterator} concept is insufficiently distinguished from
\tcode{Sentinel} to deserve a separate named concept. The current design pushes
\tcode{Iterator}'s \tcode{Sentinel<I,I>()} requirement up into
\tcode{ForwardIterator} so that the semantics of forward and stronger iterators
are unchanged, and pushes the \tcode{EqualityComparable} requirement from
\tcode{Iterator} down into \tcode{Sentinel}, but only requires it when the
iterator and sentinel types are the same. \tcode{Iterator} may then be struck
from the design, and the \tcode{Weak} prefix stripped from \tcode{WeakIterator}
as well.

\pnum
The end result of the design changes is to simplify the iterator concepts
overall, resulting in a reduction of the number of iterator categories from
N4560's seven to five as in \Cpp{}14.

\rSec1[sentrework.conc]{Conclusions}

\pnum
The Ranges TS originally inherited the notions of distinct ``weak'' and
``strong'' kinds of iterators from the Palo Alto report~(\cite{palo-alto}).
The distinction was necessary in the Palo Alto design to capture the difference
between a lone iterator that need not support equality comparison and iterator
pairs that must support equality comparison to be capable of denoting ranges.
The addition of sentinels to the TS design makes that distinction unnecessary:
iterators denote elements, ranges are denoted by iterator / sentinel pairs. A
homogeneous pair of iterators can denote a range by satisfying the
\tcode{Sentinel} concept, there's no need for separate ``weak'' and
``strong'' iterator concepts muddying the waters.
